{\bfseries CMA} (Chinese Morphological Analyzer) is a platform-\/independent C++ library for Chinese word segmentation and POS tagging. And Chinese Morphological Analyzer(Chen) is using Maxent Model and Character-\/baed Segmentation to perform segmentation and POS tagging.\section{Compile the file}\label{index_compilefile}
Two approaches can be used to compile. For the final user, the second approach is recommended.

1. Use the cmake directly.


\begin{DoxyEnumerate}
\item Using shell, go to the project root directory. 
\item Type ".cd build ". 
\item Under linux, type "cmake ../source"; Under windows, run in the msys, type "cmake -\/G "Unix Makefiles" ../source ". 
\item Finally Type ".make ". to compile all the source. 
\end{DoxyEnumerate}

2. Use a script build/build.sh to compile. The usage is:\par
 {\bfseries ./build.sh [ release $|$ debug $|$ profile $|$ clean ] [ linux $|$ win32 ]}\par


The parameters contains four building types (default is {\bfseries release}) and two platforms (default is {\bfseries linux}). For the building type, {\bfseries debug} is compiled with debug information; {\bfseries release} is for release purpose with optimized code; {\bfseries profile} will add -\/pg which used by gprof; and {\bfseries clean} will clean all temporal files. For the platform, {\bfseries linux} represents linux platform while {\bfseries win32} represents windows platform.

If the external program uses the library, simply add all the header files in the \$include\$ directory under the project root directory, and add the lib/libcmac.a into the library for static linking or lib/$\ast$.so for dynamic linking.\section{Run the Trainer}\label{index_runtrainer}
The dataset have to be trained by the Trainer. The Trainer is a executable file with name camctrainer under directory bin.

The SYNOPSIS for the trainer is: \par
 ./cmactrainer pocXmlFile mateFile modelPath [encoding] [posDelimiter] [pos $|$ poc $|$ pos+poc] [large-\/corpus] \par


\par
The Description for the parameters: 
\begin{DoxyItemize}
\item pocXmlFile is the path of Segmentation XML file in the associated encoding. One sample is db/icwb/utf8/poc.xml (With utf8 encoding).  
\item mateFile is the material file, the words separated by spaces and maybe has POS information. For example, mateFile with POS information: word1/pos1 word2/pos2...; mateFile without POS information: word1 word2.... 
\item modelPath is the directory to hold all the output files (include trained model files and dictionaries). 
\item encoding is the encoding of the mateFile, Default value is gb18030. Support utf8, gb2312, gb18030 and big5 now. Recommend to use utf8/gb18030 to support both Simplified and Traditional Chinese. 
\item posDelimiter is the delimiter between the word and the pos tag. Default value is '/'.Like '/' and '\_\-'. If the taining model is poc(the next parameter) and the mateFile doesn't contains POS information, posDemiliter can be any single character (recommend to ues '/'). 
\item pos $|$ poc $|$ pos+poc is the training mode. Default value is pos+poc. poc only trains segmentation model, pos only trains POS tagging model, and pos+poc trains both of them. 
\item large-\/corpus indicates whether the mateFile is large ( $>$ 20MB). Default value in none (don't add parameter large-\/corpus). some parameters are adjusted for large mateFile. 
\end{DoxyItemize}

Take "/dir1/dir2 " as the cateFile, after the training. The following files are created (All under directory /dir1/dir2): 
\begin{DoxyEnumerate}
\item {\bfseries pos.model} is the POS statistical model file. Exists when training mode is pos or pos+poc. 
\item {\bfseries pos.pos} is the all the POS gained from the training dataset. Exists when training mode is pos or pos+poc. 
\item {\bfseries sys.dic} is the dictionary (include words and POS tags) gained from the training dataset. This file is plain text and should be loaded as user dictionary. To convert it to the system dictionary, use Knowledge::encodeSystemDict(const char$\ast$ txtFileName, const char$\ast$ binFileName), then the binFileName can be loaded as the system dictioanry. Exists when training mode is pos or pos+poc. 
\item {\bfseries poc.model} is the POC statistical model file. Exists when training mode is poc or pos+poc. 
\end{DoxyEnumerate}

All the files are required to run the program.\par


{\bfseries Notice:} Configuration files ( cma.config, poc.xml and pos.config) are not created automatically. You can copy these files with same names under db/icwb/utf8 (If the encoding of mateFile isn't utf8, require to change the encoding of copied files).\section{Run the Demo}\label{index_rundemo}
After the training, you can run the demo to segment the file, The Demo is a executable file with name camcsegger under directory bin.

The SYNOPSIS for the trainer is: \par
 ./cmacsegger modelPath inFile outFile [encoding] [posDelimiter] \par


\par
The Description for the parameters: 
\begin{DoxyItemize}
\item modelPath is the directory contains all the tained models, dictionaries and configuration.  
\item inFile the input file. 
\item outFile the output file. 
\item encoding is the encoding of the mateFile, and gb2312 is the default encoding. Support utf8, gb18030, gb2312 and big5. 
\item posDelimiter is the delimiter between the word and the pos tag, like '/' and '\_\-' and default is '/'. 
\end{DoxyItemize}

The result with pos tagging can be found in the outFile.\section{Use the API}\label{index_useapi}
First of the all, add the lib/libcmac.a into the library path.

{\itshape Step 1: Include the header files in directory \char`\"{}include\char`\"{}\/} 
\begin{DoxyCode}
#include "cma_factory.h"
#include "analyzer.h"
#include "knowledge.h"
#include "sentence.h"
\end{DoxyCode}


{\itshape Step 2: Use the library name space\/} 
\begin{DoxyCode}
using namespace cma;
\end{DoxyCode}


{\itshape Step 3: Call the interface and handle the result\/}


\begin{DoxyCode}
// create instances
CMA_Factory* factory = CMA_Factory::instance();
Analyzer* analyzer = factory->createAnalyzer();
Knowledge* knowledge = factory->createKnowledge();

//It is suggested to set encoding after crate the Knowledge. Another supported en
      code type is big5.
knowledge->setEncodeType(Knowledge::ENCODE_TYPE_GB2312);

// Load all the model by specific modelPath and encoding
knowledge->loadModel( "gb18030", "..." ).

// Load User Dictionaries and Stop Words if neccessary.
knowledge->loadUserDict("...");
knowledge->loadStopWordDict("...");
 
// (optional) if POS tagging is not needed, call the function below to turn off t
      he analysis and 
// output for POS tagging, so that large execution time could be saved when execu
      te 
// Analyzer::runWithSentence(), Analyzer::runWithString(), Analyzer::runWithStrea
      m().
analyzer->setOption(Analyzer::OPTION_TYPE_POS_TAGGING, 0);

// (optional) set the number of N-best results,
// if this function is not called, one-best analysis is performed defaultly on An
      alyzer::runWithSentence().
analyzer->setOption(Analyzer::OPTION_TYPE_NBEST, 5);

// set knowledge
analyzer->setKnowledge(knowledge);

// 1. analyze a paragraph
const char* result = analyzer->runWithString("...");
...

// 2. analyze a file
analyzer->runWithStream("...", "...");

// 3. split paragraphs into sentences
string line;
vector<Sentence> sentVec;
while(getline(cin, line)) // get paragraph string from standard input
{
    sentVec.clear(); // remove previous sentences
    analyzer->splitSentence(line.c_str(), sentVec);
    for(size_t i=0; i<sentVec.size(); ++i)
    {
        analyzer->runWithSentence(sentVec[i]); // analyze each sentence
        ...
    }
}

// destroy instances
delete knowledge;
delete analyzer;
\end{DoxyCode}
 